\documentclass{article}

    % Input language encoding
    \usepackage[utf8]{inputenc}
   
    % Output languages
    \usepackage[english, greek]{babel}
    \usepackage{alphabeta}
    
    % Fonts
    \usepackage[T1,LGR]{fontenc}
    \usepackage{lmodern}

    % Images
    \usepackage{graphicx}
    \usepackage{float}
    \usepackage{caption}
    \usepackage{subcaption}

    % Math
    \usepackage{amsmath}
    \usepackage{amssymb}

    % Paragraph Formatting
    \usepackage{parskip}

    % Code
    \usepackage{listings}
    \usepackage{fancyvrb}

    % Different Enumerations
    \usepackage{enumitem}

    % Trees
    \usepackage{qtree}

    % Other Drawings
    \usepackage{tikz}
    \usetikzlibrary{shapes,backgrounds}

    % Links
    \usepackage{hyperref}

    % Color
    \usepackage{color}
   
    % Setup

    % For hyperlinks
    \hypersetup{
        colorlinks=true,
        linkcolor=blue,
        filecolor=magenta,      
        urlcolor=cyan,
    }

    \urlstyle{same}
    
    % For code
    \definecolor{codegreen}{rgb}{0,0.6,0}
    \definecolor{codegray}{rgb}{0.5,0.5,0.5}
    \definecolor{codepurple}{rgb}{0.58,0,0.82}
    \definecolor{backcolour}{rgb}{0.95,0.95,0.92}
     
    \lstdefinestyle{mystyle}{
        backgroundcolor=\color{backcolour},   
        commentstyle=\color{codegreen},
        keywordstyle=\color{magenta},
        numberstyle=\tiny\color{codegray},
        stringstyle=\color{codepurple},
        basicstyle=\fontsize{8}{11}\selectfont\ttfamily,
        breakatwhitespace=false,         
        breaklines=true,                 
        captionpos=b,                    
        keepspaces=true,                 
        numbers=left,                    
        numbersep=5pt,                  
        showspaces=false,                
        showstringspaces=false,
        showtabs=false,                  
        tabsize=4
    }

    \lstset{style=mystyle}

    % For math
    \DeclareMathSizes{10}{10}{10}{10}
    \setlength{\parindent}{0cm}

    % Foreign Language macro
    \newcommand{\english}[1]{\foreignlanguage{english}{{#1}}}

    \title{2η Εργαστηριακή Ασκήσεων \\
        Σχεδιασμός Ενσωματωμένων Συστημάτων}
\begin{document}

\pagenumbering{arabic}
\date{}
\author{Λεωνίδας Αβδελάς $|$ ΑΜ: 03113182}

\maketitle
\newpage

\section*{Άσκηση 1: Βελτιστοποίηση δυναμικών δομών δεδομένων \\ του αλγορίθμου \english{DRR}}

Οι συνδιασμοί των υλοιποιήσεων δομών δεδομένω και τα αποτελέσματα φαίνονται στον πίνακα:

\begin{otherlanguage}{english}
    \begin{center}
        \begin{tabular}{|p{100pt}|c|c|}\hline
            & Memory accesses & Memory footprint \\ \hline
            Nodes: SLL, \newline  Package: SLL & 70820207 & 798.8KB \\ \hline
            Nodes: DLL, \newline Package: SLL & 70832593 & 823.0KB \\ \hline
            Nodes: DLL, \newline Package: DLL & 71471605 & 983.3KB \\ \hline
            Nodes: SLL, \newline Package: DLL & 71459376 & 980.3KB \\ \hline
            Nodes: DYN ARR, \newline Package: DLL & 71966451 & 928.5KB \\ \hline
            Nodes: DYN ARR, \newline Package: SLL & 71288307 & 760.2KB \\ \hline
            Nodes: DYN ARR, \newline Package: DYN ARR & 472028393 & 1.075MB \\ \hline
            Nodes: SLL, \newline Package: DYN ARR & 471333547 & 1.111MB \\ \hline
            Nodes: DLL, \newline Package: DYN ARR & 471350334 & 1.128MB \\ \hline
       \end{tabular}
    \end{center}
\end{otherlanguage}

Οπως βλέπουμε, ο συνδιασμός υλοποιήσεων που έχει τις \textbf{λιγότερες προσβάσεις} στην μνήμη είναι με τη χρήση \textbf{\english{SLL} και για τους κόμβους και για τα πακέτα}.

Για το \english{\textbf{memory footprint}} η καλύτερη επιλογή είναι \textbf{δυναμικός πίνακας για τους κόμβους και απλή λίστα για τα πακέτα}.

\section*{Άσκηση 2: Βελτιστοποίηση δυναμικών δομών δεδομένων \\ του αλγορίθμου \english{Dijkstra}}

Τα αποτελέσματα για τις διαφορετικές δομές δεδομένων φαίνονται παρακάτω:

\begin{otherlanguage}{english}
    \begin{center}
        \begin{tabular}{|p{100pt}|c|c|}\hline
            & Memory accesses & Memory footprint \\ \hline
            SLL & 102900357 & 359.2KB \\ \hline
            DLL & 103081674 & 473.7KB \\ \hline
            Dynamic Array & 150559543 & 363.7KB \\ \hline
       \end{tabular}
    \end{center}
\end{otherlanguage}

Οπως βλέπουμε, η υλοποίηση που έχει τις \textbf{λιγότερες προσβάσεις} στην μνήμη είναι η  \english{\textbf{SLL}}.

Για το \english{\textbf{memory footprint}} η καλύτερη επιλογή είναι ξανά η \textbf{απλή λίστα}.

\end{document}