\documentclass{article}

    % Input language encoding
    \usepackage[utf8]{inputenc}
   
    % Output languages
    \usepackage[english, greek]{babel}
    \usepackage{alphabeta}
    
    % Fonts
    \usepackage[T1,LGR]{fontenc}
    \usepackage{lmodern}

    % Images
    \usepackage{graphicx}
    \usepackage{float}
    \usepackage{caption}
    \usepackage{subcaption}

    % Math
    \usepackage{amsmath}
    \usepackage{amssymb}

    % Paragraph Formatting
    \usepackage{parskip}

    % Code
    \usepackage{listings}
    \usepackage{fancyvrb}

    % Different Enumerations
    \usepackage{enumitem}

    % Trees
    \usepackage{qtree}

    % Other Drawings
    \usepackage{tikz}
    \usetikzlibrary{shapes,backgrounds}

    % Links
    \usepackage{hyperref}

    % Color
    \usepackage{color}
   
    % Setup

    % For hyperlinks
    \hypersetup{
        colorlinks=true,
        linkcolor=blue,
        filecolor=magenta,      
        urlcolor=cyan,
    }

    \urlstyle{same}
    
    % For code
    \definecolor{codegreen}{rgb}{0,0.6,0}
    \definecolor{codegray}{rgb}{0.5,0.5,0.5}
    \definecolor{codepurple}{rgb}{0.58,0,0.82}
    \definecolor{backcolour}{rgb}{0.95,0.95,0.92}
     
    \lstdefinestyle{mystyle}{
        backgroundcolor=\color{backcolour},   
        commentstyle=\color{codegreen},
        keywordstyle=\color{magenta},
        numberstyle=\tiny\color{codegray},
        stringstyle=\color{codepurple},
        basicstyle=\fontsize{8}{11}\selectfont\ttfamily,
        breakatwhitespace=false,         
        breaklines=true,                 
        captionpos=b,                    
        keepspaces=true,                 
        numbers=left,                    
        numbersep=5pt,                  
        showspaces=false,                
        showstringspaces=false,
        showtabs=false,                  
        tabsize=4
    }

    \lstset{style=mystyle}

    % For math
    \DeclareMathSizes{10}{10}{10}{10}
    \setlength{\parindent}{0cm}

    % Foreign Language macro
    \newcommand{\english}[1]{\foreignlanguage{english}{{#1}}}

    \title{1η Εργαστηριακή Ασκήσεων \\
        Σχεδιασμός Ενσωματωμένων Συστημάτων}
\begin{document}

\pagenumbering{arabic}
\date{}
\author{Λεωνίδας Αβδελάς $|$ ΑΜ: 03113182}

\maketitle
\newpage

\section*{Ζητούμενο 1ο - \english{Loop Optimizations \\ \& Design Space Exploration}}

\subsection*{1.}
\begin{itemize}
    \item Για την έκδοση λειτουργικού και την έκδοση πυρήνα \english{Linux}, θα χρησιμοποιήσουμε τις εντολές \english{\lstinline[language=bash]{cat /etc/os-release}} και \english{\lstinline[language=bash]{uname -r}}, αντίστοιχα. Τα αποτελέσματα είναι \english{PRETTY\_NAME=Debian GNU/Linux 10 (buster)} και \english{4.19.0-6-amd64}.
    \item Για την ιεραρχία μνήμης, χρησιμοποιήσαμε την εντολή \english{\lstinline[language=bash]{sudo lshw -C memory}}. Τα αποτελέσματα φαίνονται στον παρακάτω πίνακα:\\
    \begin{otherlanguage}{english}
        \begin{center}
            \begin{tabular}{|c|c|c|c|c|}\hline
                & L1 cache & L2 cache & L3 cache & RAM \\ \hline
                Size & 32KiB & 256KiB & 3MiB & 8GiB \\ \hline
           \end{tabular}
        \end{center}
        \end{otherlanguage}
    \item Τις πληροφορίες για τους πυρήνες θα τις βρούμε στο αρχείο \english{/proc/cpuinfo}. Έτσι, για τον αριθμό των πυρήνων, τρέχουμε:
    \\ \english{\lstinline[language=bash]{cat /proc/cpuinfo \| grep processor \| wc -l}} \\
    από όπου παίρνουμε την απάντηση 4 και για την ταχύτητα τους, τρέχουμε :\\
    \english{\lstinline[language=bash]{cat /proc/cpuinfo \|  grep 'cpu MHz'}} \\
    από όπου παίρνουμε την απάντηση \english{~800Mhz}.
\end{itemize}

\subsection*{2.}

Προσθέτοντας τον υπολογισμό χρόνου στο πρόγραμμα μας και τρέχοντας το 10 φορές, έχουμε τα παρακάτω αποτελέσματα: \\
\begin{otherlanguage}{english}
    \begin{center}
        \begin{tabular}{|c|c|c|c|}\hline
            & Average & Maximum & Minimum \\ \hline
            Time &  12129.0ms & 15032ms & 11332ms\\ \hline
       \end{tabular}
    \end{center}
    \end{otherlanguage}

    Ένα πράγμα που παρατηρούμε ότι η διαδοχικές εκτελέσεις του προγράμματος επωφελούνται από το \english{caching} των δεδομένων. Έτσι, η πρώτη φορά που τρέχει το πρόγραμμα απαιτεί τον μέγιστο χρόνο, και οι επόμενες απαιτούν αισθητά λιγότερο χρόνο. 

\subsection*{3.}

Εξατάζουμε τον κώδικα, η πρώτη αλλαγή που βλέπουμε ότι μπορεί να γίνει είναι ένα \english{loop merging} των loops για τον άξονα x. Έτσι γλυτώνουμε \english{B} επαναλήψεις για κάθε \english{pixel}.
\end{document}